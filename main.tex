\documentclass{report}
\title{Proposition de th\`ese}
\author{Yannick Hold-Geoffroy  \\
    Universit\'e laval  \\
    }

\date{\today}

% Hint: \title{what ever}, \author{who care} and \date{when ever} could stand 
% before or after the \begin{document} command 
% BUT the \maketitle command MUST come AFTER the \begin{document} command! 
\begin{document}

\maketitle


\begin{abstract}
Short introduction to subject of the paper \ldots 
\end{abstract}

\section{Introduction}

\subsection{Contexte de la recherche}


\subsection{Vision future}


\section{Description du projet}

La Stéréoscopie Photométrique (SP) est une technique permettant de récupérer la structure d'une scène à partir d'indices photométriques.

[Image ]

\subsection{Stéréoscopie Photométrique}

\subsection{Contraintes extérieures}


\section{Revue de littérature}

\cite{Hold-Geoffroy-3DV15,Hold-Geoffroy-ICCP15}


\section{Solutions}

\subsection{Modèles d'illumination plus riche}

\subsection{Emploi d'autres astres que le soleil}

\subsection{Augmentation avec d'autres techniques}


\section{Échéancier}


\section{Conclusion}\label{conclusion}


{\small
%\bibliographystyle{ieee}
\bibliography{main}
}

\end{document}